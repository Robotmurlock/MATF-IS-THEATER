% !TEX encoding = UTF-8 Unicode
\documentclass[a4paper]{article}

\usepackage{color}
\usepackage[obeyspaces]{url}
\usepackage[T1]{fontenc} % enable Cyrillic fonts
\usepackage[utf8]{inputenc} % make weird characters work
\usepackage{graphicx}
% \usepackage[margin=1.5in]{geometry}
\usepackage[table]{xcolor}
\usepackage{tocloft}
\usepackage{float}
\addtolength{\cftsecnumwidth}{10pt}
\usepackage{wasysym}

\usepackage[english,serbian]{babel}
%\usepackage[english,serbianc]{babel} 

\PassOptionsToPackage{obeyspaces}{url}
\usepackage[unicode]{hyperref}
\hypersetup{colorlinks,citecolor=green,filecolor=green,linkcolor=blue,urlcolor=blue}

\usepackage{listings}
\renewcommand\lstlistingname{Kod}
\renewcommand\lstlistlistingname{Kodovi}

\definecolor{mygreen}{rgb}{0,0.6,0}
\definecolor{mygray}{rgb}{0.5,0.5,0.5}
\definecolor{mymauve}{rgb}{0.58,0,0.82}

\lstset{
  backgroundcolor=\color{white},   % choose the background color; you must add \usepackage{color} or \usepackage{xcolor}; should come as last argument
  basicstyle=\scriptsize\ttfamily,        % the size of the fonts that are used for the code
  breakatwhitespace=false,         % sets if automatic breaks should only happen at whitespace
  breaklines=true,                 % sets automatic line breaking
  captionpos=b,                    % sets the caption-position to bottom
  commentstyle=\color{mygreen},    % comment style
  deletekeywords={...},            % if you want to delete keywords from the given language
  escapeinside={\%*}{*)},          % if you want to add LaTeX within your code
  extendedchars=true,              % lets you use non-ASCII characters; for 8-bits encodings only, does not work with UTF-8
  firstnumber=1,                   % start line enumeration with line 1000
  frame=single,	                   % adds a frame around the code
  keepspaces=true,                 % keeps spaces in text, useful for keeping indentation of code (possibly needs columns=flexible)
  keywordstyle=\color{blue},       % keyword style
  language=Python,                 % the language of the code
  morekeywords={*,...},            % if you want to add more keywords to the set
  numbers=left,                    % where to put the line-numbers; possible values are (none, left, right)
  numbersep=5pt,                   % how far the line-numbers are from the code
  numberstyle=\tiny\color{mygray}, % the style that is used for the line-numbers
  rulecolor=\color{black},         % if not set, the frame-color may be changed on line-breaks within not-black text (e.g. comments (green here))
  showspaces=false,                % show spaces everywhere adding particular underscores; it overrides 'showstringspaces'
  showstringspaces=false,          % underline spaces within strings only
  showtabs=false,                  % show tabs within strings adding particular underscores
  stepnumber=2,                    % the step between two line-numbers. If it's 1, each line will be numbered
  stringstyle=\color{mymauve},     % string literal style
  tabsize=2,	                     % sets default tabsize to 2 spaces
  title=\lstname                   % show the filename of files included with \lstinputlisting; also try caption instead of title
}

\begin{document}

\title{Informacioni sistem pozorišta\\ \small{Seminarski rad u okviru kursa\\Informacioni sistemi\\ Matematički fakultet}}

\author{
Katarina Savičić 1086/2020\\
Dragana Milić 1042/2019\\
Nikola Vuković 1090/2020\\
Ognjen Milinković 1008/2020\\
Momir Adžemović 1005/2020
}

\date{1.~april 2020.}

\maketitle

\abstract

Ovo je sažetak

\newpage

\tableofcontents

\newpage

\section{Uvod}
Ovo je uvod.

\section{Slučajevi upotrebe.}

\subsection{Organizacija predstave}
Da bi se predstava odrzala, potrebno je da korisnik koji je registrovan kao organizator kreira predstavu. U IS se belezi predstava sa stanjem “prijava”. Supervizor pozorista moze da odobri ili odbije prijavljenu predstavu. Ako se prijava odobri, prelazi u sledece stanje odobrena. Korisnik koji je registrovan kao glumac moze da se prijavi za jednu (ili vise) uloga. Svaka odobrena predstava ima ulogu rezerva, za koju moze da se prijavi bilo koji glumac. Supervizor moze da promeni stanje predstave koja ima stanje odobrena i ima popunjena mesta glumca kao u pripremi. 

\subsection{Rezervisanje pozorišnih sala}
Reditelj predstave u pripremi treba da rezerviše salu za probe. Održavanje predstave takođe mora da ima salu. Dakle treba omogućiti uvid u slobodne sale, kao i rezervacije unapred.

\subsection{Održavanje predstave}
Predstava sa statusom “u pripremi” može da se zakaže za održavanje određenog datuma u određenoj sali. 

\subsection{Trgovine kartama}
Korisnik koji je registovan kao posetilac može da kupi kartu za održavanje predstave ako postoji slobodno mesto.  
Supervizor i organizator mogu da vide slobodna/zauzeta mesta za održavanje predstave

\subsection{Vođenje finansija}
Administrator pozorišta može da vidi prilive i odlive. Ovo takođe može da se poveže sa prijavom predstave koja može da zahteva određen honorar za glumce


\section{Druga sekcija.}
Ovo je druga sekcija.

\section{Treća sekcija.}
Ovo je treća sekcija.

\section{Zaključak}
Ovo je zaključak.

\newpage

\addcontentsline{toc}{section}{Literatura}
\appendix
\bibliography{literatura} 
\bibliographystyle{unsrt}

\end{document}
        