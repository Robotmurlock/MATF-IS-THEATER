% !TEX encoding = UTF-8 Unicode
\documentclass[a4paper]{article}

\usepackage{color}
\usepackage[obeyspaces]{url}
\usepackage[T1]{fontenc} % enable Cyrillic fonts
\usepackage[utf8]{inputenc} % make weird characters work
\usepackage{graphicx}
% \usepackage[margin=1.5in]{geometry}
\usepackage[table]{xcolor}
\usepackage{tocloft}
\usepackage{float}
\addtolength{\cftsecnumwidth}{10pt}
\usepackage{wasysym}

\usepackage[english,serbian]{babel}
%\usepackage[english,serbianc]{babel} 

\PassOptionsToPackage{obeyspaces}{url}
\usepackage[unicode]{hyperref}
\hypersetup{colorlinks,citecolor=green,filecolor=green,linkcolor=blue,urlcolor=blue}

\usepackage{listings}
\renewcommand\lstlistingname{Kod}
\renewcommand\lstlistlistingname{Kodovi}

\definecolor{mygreen}{rgb}{0,0.6,0}
\definecolor{mygray}{rgb}{0.5,0.5,0.5}
\definecolor{mymauve}{rgb}{0.58,0,0.82}

\lstset{
  backgroundcolor=\color{white},   % choose the background color; you must add \usepackage{color} or \usepackage{xcolor}; should come as last argument
  basicstyle=\scriptsize\ttfamily,        % the size of the fonts that are used for the code
  breakatwhitespace=false,         % sets if automatic breaks should only happen at whitespace
  breaklines=true,                 % sets automatic line breaking
  captionpos=b,                    % sets the caption-position to bottom
  commentstyle=\color{mygreen},    % comment style
  deletekeywords={...},            % if you want to delete keywords from the given language
  escapeinside={\%*}{*)},          % if you want to add LaTeX within your code
  extendedchars=true,              % lets you use non-ASCII characters; for 8-bits encodings only, does not work with UTF-8
  firstnumber=1,                   % start line enumeration with line 1000
  frame=single,	                   % adds a frame around the code
  keepspaces=true,                 % keeps spaces in text, useful for keeping indentation of code (possibly needs columns=flexible)
  keywordstyle=\color{blue},       % keyword style
  language=Python,                 % the language of the code
  morekeywords={*,...},            % if you want to add more keywords to the set
  numbers=left,                    % where to put the line-numbers; possible values are (none, left, right)
  numbersep=5pt,                   % how far the line-numbers are from the code
  numberstyle=\tiny\color{mygray}, % the style that is used for the line-numbers
  rulecolor=\color{black},         % if not set, the frame-color may be changed on line-breaks within not-black text (e.g. comments (green here))
  showspaces=false,                % show spaces everywhere adding particular underscores; it overrides 'showstringspaces'
  showstringspaces=false,          % underline spaces within strings only
  showtabs=false,                  % show tabs within strings adding particular underscores
  stepnumber=2,                    % the step between two line-numbers. If it's 1, each line will be numbered
  stringstyle=\color{mymauve},     % string literal style
  tabsize=2,	                     % sets default tabsize to 2 spaces
  title=\lstname                   % show the filename of files included with \lstinputlisting; also try caption instead of title
}

\begin{document}

\title{Informacioni sistem pozorišta\\ \small{Seminarski rad u okviru kursa\\Informacioni sistemi\\ Matematički fakultet}}

\author{
Katarina Savičić 1086/2020\\
Dragana Milić 1042/2019\\
Nikola Vuković 1090/2020\\
Ognjen Milinković 1008/2020\\
Momir Adžemović 1005/2020
}

\date{1.~april 2020.}

\maketitle

\abstract

Ovo je sažetak.

\newpage

\tableofcontents

\newpage

\section{Uvod}
Ovo je uvod.

\section{Analiza sistema}


\section{Slučajevi upotrebe.}

\subsection{Organizacija predstave}
\begin{enumerate}
  \item \textbf{Kratak opis:} Organizator predstave je zadužen za ceo proces formiranja predstave od njene prijave do realizacije. 
        Proces formiranja podrazumeva prijavu koju mora da odobri supervizor pozorišta, unajmljivanje 
        glumaca i rezervacija sala za predstave. Neophodno je da organizator konstantno ima uvid u stanje
        svake predstave za koji je on zadužen.
  \item \textbf{Učesnici:} \dots
  \item \textbf{Preduslovi:} \dots
  \item \textbf{Postuslovi:} \dots
  \item \textbf{Osnovni tok:} \dots
  \item \textbf{Alternativni tokovi:} \dots
  \item \textbf{Podtokovi:} \dots
  \item \textbf{Specijalni zahtevi:} \dots
  \item \textbf{Dodatne informacije:} \dots
% Ovo sve može u tok
% Da bi se predstava održala, potrebno je da organizator napravi predstavu u okviru IS-a. 
% Predstava ima inicijalno stanje \textit(prijava). Supervizor pozorišta može da odobri ili 
% odbije prijavljenu predstavu. Ako se prijavljena predstava odobri, onda predstava prelazi 
% u stanje \textit{odobrena}. Glumac može da se prijavi za jednu (ili više) uloga za predstavu koja je
% odobrena. Svaka odobrena predstava ima ulogu rezerva, za koju može da se prijavi bilo koji glumac. 
% Supervizor može da promeni stanje predstave koja ima stanje odobrena i ima popunjena mesta glumca 
% kao u pripremi. 
\end{enumerate}

\subsection{Rezervacija pozorišnih sala}
\begin{enumerate}
  \item \textbf{Kratak opis:} Organizator predstave u pripremi treba da rezerviše salu za probu, 
        audiciju ili konkretnu predstavu. Dakle, treba omogućiti uvid u slobodne sale, 
        kao i rezervacije unapred.
\end{enumerate}

\subsection{Audicije i konkursi za posao}
\begin{enumerate}
  \item \textbf{Kratak opis:} Konkursi mogu da podrazumevaju zapošljavanje za stalno ili privremeno
        (jedna ili više predstava). Konkursi mogu da se odnose na različite uloge kao što su
        organizator, glumac, itd... Audicije se odnose konkretno za glumce.
\end{enumerate}
 
\subsection{Kupovina i rezervacija karata}
\begin{enumerate}
  \item \textbf{Kratak opis:} Informacioni sistem nudi pregled repertoara. 
        Svako može da kupi kartu za određenu predstavu ako postoji slobodno mesto.
        Alternativa je da se vrši rezervacija karte koja se kasnije može da kupi u blagajni. 
\end{enumerate}

\subsection{Vođenje finansija}
\begin{enumerate}
  \item \textbf{Kratak opis:} Supervizor pozorišta može da vidi prilive i odlive za ceo sistem ili
        neku konkretnu predstavu. 
\end{enumerate}

\section{Druga sekcija.}
Ovo je druga sekcija.

\section{Treća sekcija.}
Ovo je treća sekcija.

\section{Zaključak}
Ovo je zaključak.

\newpage

\addcontentsline{toc}{section}{Literatura}
\appendix
\bibliography{literatura} 
\bibliographystyle{unsrt}

\end{document}
        